\section{Une expérience à 20mK}
Les éxpériences actuelles de l’équipe n’utilisent ni des molécules de Fulerène, ni un testeur sous pointes et n’ont pas lieu à 4,2K. De plus le convertisseur courant-tension n’est pas un modèle commercial mais a été fabriqué au CNRS pour permettre des mesures ultra bas bruit.

En effet la molécule de Fulerène n’ayant aucune propriété particulière elle a été rapidement remplacé par des molécules aux propriétés plus intéressantes, notamment les aimants moléculaires dont nous parlions en introduction. Par exemple TbPc2 dans le cas de l’équipe de nanospintronique et transport moléculaire.
Les contacts par pointes sont aussi remplacée par des micro-soudures qui possèdent plusieurs avantages sur le contact par pointes. Bien qu’une telle technique soit plus délicate et plus longue à mettre en place et ne permet de tester les transistors qu’en faible nombre, elle a comme intérêts majeurs de réduire le bruit, et permettre un contact parfait sur des durées très longues (certaines expériences sont en cours depuis plus d’un an)

\begin{figure}[h]
    \begin{center}
       % \includegraphics[width=150px]{Photos/Refrigerateur_Dillution.png}
        \caption{Photographie du réfrigérateur à dilution et de l’ensemble du banc de mesure}
        \label{fig:}
    \end{center}
\end{figure}
Photographie disponible à https://tel.archives-ouvertes.fr/tel-00984973/document

Pour finir l’expérience actuelle se déroule actuellement à des températures de 30mK dans un réfrigérateurs à dilution, la température de électronique étant elle de l’ordre de 80mK \cite{10}. Le principe du réfrigérateur à dilution est comparable à celui d’un réfrigérateur classique, des détentes de Joule-Thomsom successives pour refroidir l’hélium à de très basses températures. Le réfrigérateur étant constitué de différents étages dont les températures vont en diminuant pour limiter les pertes. Il faut cependant attendre plusieurs heures pour qu’une température de 30mK soit atteinte.
\section{Vers l'informatique quantique?}
